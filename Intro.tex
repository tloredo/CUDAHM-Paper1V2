%===============================================================================
\section{Introduction}
\label{sec:intro}

Bayesian inference with graphical models has rapidly grown in popularity and sophistication since the emergence of Markov chain Monte Carlo (MCMC) algorithms for Bayesian computation nearly three decades ago.
The work we report here focuses on models with classic, simple graphical structures---directed acyclic graphs (DAGs) with a single plate, i.e., a single level of replication of random variables at the lower level of a hierarchical model.
Our work aims to extend the range of application of Bayesian graphical modeling in the direction of increased dataset size, rather than in the direction of increased graphical complexity.

We are motivated by measurement error problems in astronomy: density estimation with measurement error (density deconvolution, or demixing, often of a \emph{conditional} density), and linear and nonlinear regression with measurement errors in both predictors and response.
Hierarchical Bayesian modeling is well-suited to such problems, but is relatively new in astronomy (see \citealt{loredo2013survey} for a recent survey).
Although some recent astrostatistical research develops models with rich graphical structure, most astronomers are unfamiliar with hierarchical modeling, and models with simple graphical structure can provide new capability in many areas of astronomy, including basic two-level hierarchical models.%
\footnote{We follow the convention of naming hierarchical DAGs by the number of levels with uncertain nodes, e.g., the number of open nodes in Fig.~\ref{fig:DAG-2Level}, which depicts a model with three levels of random variables, but with the data variables (bottom level) known, i.e., to be conditioned on.}
But dataset size can be an obstacle to use of such models.
Large-scale, automated surveys are providing astronomers with increasingly large datasets for demographic studies of cosmic populations (e.g., categories of stars, galaxies, and planets).
Current and emerging surveys are providing measurements for populations with sizes ranging from tens of thousands to $10^8$ or even larger.
For datasets of these scales, exploration or integration over the latent variables specifying imprecisely measured characteristics of objects in a population can be prohibitive, even for simple models with univariate member characteristics.
Yet as population size grows, it becomes increasingly important to account for uncertainty in such latent variables.
For example, it is well known that regression and density estimators that ignore measurement error are typically inconsistent, with the ratio of bias to reported precision growing with sample size \citep{C+06-MsmtErr}.
Single-plate hierarchical models can account for measurement error in many astronomically interesting scenarios, provided the implementation enables efficient computation for relevant dataset sizes.

%Section~\ref{sec:app_models_methods} describes the applied models and methods in general.

In the following section (Sec.~\ref{sec:design}), we describe the architecture of the CUDAHM framework, which is motivated by a common computational structure underlying example hierarchical models arising in measurement error problems in astronomy. 
%\enote{Perhaps add a brief section (\S~3) treating Brandon's nonlinear regression with measurement error problem; perhaps he'd be willing to write it; a very brief summary of it is in \S~2 and may suffice, with Brandon's example presented in the followup A\&C paper.}
In Sec.~\ref{sec:lum_func}, we describe a common astronomical data analysis problem: inferring the luminosity distribution of a class of objects from distance and flux observations of a sample subject to selection effects and measurement error.
Such problems may be modeled using latent, thinned marked point processes observed with measurement error; we show that the resulting likelihood function can be cast in a form mirroring the computational structure of single-plate graphical models, enabling implementation with CUDAHM.
We present tests of such an implementation, using simulated data, in Sec.~\ref{sec:app_on_sim_data}.
Sec.~\ref{sec:summary} provides a summary and plans for future work.
In the supplementary material, Sec.~\ref{sec:comp_mu_theta} presents an approximation technique for the computation of an integral appearing in the luminosity function example.
